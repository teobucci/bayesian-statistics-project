%!TEX root = ./main.tex

\section{Introduction}

%First paragraph introducing GGM
Gaussian graphical models (GGM) are probabilistic models where undirected graphs are used to express the conditional dependence structure among variables with a joint Gaussian distribution. The vertexes of the graph represent the variables of the model, whereas the edges model the dependency between them. A crucial concept in GGM is the one of conditional independence: given a p-dimensional random vector $\mathbf{Y}$ distributed as a multivariate normal with zero mean and precision matrix $\mathbf{K}$, the random variables $Y_i$ and $Y_j$ are independent, conditionally on all the others, if the corresponding entry in precision matrix $\mathbf{K}$ is null, owing to the normality assumption. This translates into the absence of a direct edge linking nodes $i$ and $j$ in the graph. 

The graph is usually unknown and must be learnt from the data; a common choice to perform inference is to rely on a Bayesian approach. In this case, the graph itself is considered a random variable, which can assume values in the space of all possible undirected graphs with p nodes, and requires a prior distribution to be specified. Furthermore, a prior on the precision matrix conditionally on the graph must also be introduced. A G-Wishart distribution is often selected as a prior for the precision matrix, whereas the specification of the graph prior is a more delicate step, heavily dependent on the context.

%Use SBM as a prior to induce a block structure 
In many real-life applications, the variables of a model can be grouped into disjoint clusters, resulting in a block structure of the underlying graph. For instance, several biological problems require to identify groups of genes which regulate specific cellular functions (\textbf{ ref needed}). For this purpose, Stochastic Block Models (SBM) can be used to induce a block structure in the adjacency matrix of a graph. More in detail, SBM are generative models for random networks with wide applications in the context of community detection, as they allow to cluster the nodes of a graph into mutually exclusive groups that share similar connectivity patterns and are not known a priori. Most importantly, in SBM the probability of having an edge between two nodes only depends on the group membership of the nodes. Of note, the clustering automatically induces a partition on the nodes; therefore, the space of possible graph is isomorphic to the space of possible partitions, and specifying the prior of the graph is equivalent to introduce a prior on the partition.

%Additional specification on the order of the partition - IN PROGRESS -
One additional constraint that may exist in real contexts (\textbf{should we mention /quote some of them?}) is that the the model variables present some predefined order which cannot be ignored when identifying clusters. This limits the admissible partitions to the ones where the order is preserved or, equivalently, restricts the graph space to the set of admissible graphs. 

%Introduce the goal of the project
In this project, we propose a flexible prior that allows to infere a block-structured graph while respecting an ordering constraint on the nodes. To do so, we build upon the theory of changepoint models from the works of Benson \cite{bensonAdaptiveMCMCMultiple2018} and Martinez \cite{martinezNonparametricChangePoint2014}, borrowing some ideas concerning the prior on the partition and relying on an adaptive approach. 

%Details on the structure
In the next two sections, we introduce the final model and describe the the proposed sampling strategy, which relies on a two-block Gibbs sampler to update the graph and the partition. Then, we present the main results of our simulations. Finally, we perform a critical analysis of our results and discuss the limitations of the current work, while also identifying potential directions for future investigation.  


\section{Proposed Model}


\section{Sampling Strategy}


\section{Simulation study}

La 4 invece deve contenere la descrizione degli esperimenti fatti, le figure e i risultati. Senza vincoli di spazio, anche una sbrodolata va bene.


\section{Conclusion}

Conclusioni brevi, al più riportate le criticità incontrate.