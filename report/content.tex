%!TEX root = ./main.tex



\begin{abstract} %in progress 
Gaussian graphical models are a tool to learn the conditional dependence structure among variables through the presence or absence of edges in the underlying graph. In many applications, the variables can be grouped so that the underlying graph we want to learn has a block structure. Stochastic block models offer a powerful tool to detect such structure in a network. The goal of this project is to propose a new flexible prior that accounts for a random partition of the nodes, respects their ordering constraints and allows to learn a block structured graph.
\end{abstract}


\section{Introduction}

%First paragraph introducing GGM
Gaussian graphical models (GGM) are probabilistic models where undirected graphs are used to express the conditional dependence structure among variables with joint gaussian distribution. The vertexes of the graph represent the variables of the model, whereas the edges model the dependency between them. A crucial concept in GGM is the one of conditional independence: given be a p-dimensional random vector $Y$ distributed as a multivariate normal with zero mean and precision matrix $\mathbf{K_{p}^{-1}}$, the random variables $Y_i$ and $Y_j$ are independent, conditionally on all the others, if the corresponding entry in precision matrix \tectbf{K} is null, owing to the normality assumption. This translates into the absence of a direct edge linking nodes i and j in the graph. 

In general,the graph is unknown and must be learnt from the data; a common choice to perform inference is relying on a Bayesian approach. In this case, the graph itself is considered a random variable, which can assume values in the space of all possible undirected graphs with p nodes (representing the variables of our models), and it is therefore necessary to specify a prior distrubution on it. Furthermore, a prior on the precision matrix conditionally on the graph must also be introduced. A G-Wishart distribution is often selected as a prior for the precision matrix, whereas the specification of the graph prior is a more delicate step, heavily dependent on the context.

%Use SBM as a prior to induce a block structure 
Many real life applications allow to group the variables of the model, so that the underlying graph presents a block structure. For instance, several biological problems require to identify clusters of genes which subside specific cellular functions (Maybe ref needed?). For this purpose, Stochastic Block Models (SBM) can be used to induce a block structure in the adjacency matrix of the graph. SBM are generative models for random graphs with wide applications in the context of community detection. Given a network of data represented by a graph, SBM can be used to infere the nodes partition by clustering nodes into mutually exclusive groups that share similar connectivity patterns and are not known a priori. In an SBM, the probability of having an edge between two nodes only depends on the cluster membership of the two nodes. The abovementioned clustering of the graph nodes automatically induces a partition of the variables. In light of this consideration, the prior of the graph can be alternative specified with a prior on the partition, assuming values in the space of all possible partitions.

%Additional specification on the order of the partition - IN PROGRESS -
One additional constraint that may exist in real situations (should we mention some of them??) is that the model variables present some prefefined order which cannot be ignored when identifying clusters. This limits the support of the admissible partitions to the ones where the order is preserved.     





\section{Proposed Model}


\section{Sampling Strategy}


\section{Simulation study}

La 4 invece deve contenere la descrizione degli esperimenti fatti, le figure e i risultati. Senza vincoli di spazio, anche una sbrodolata va bene.


\section{Conclusion}

Conclusioni brevi, al più riportate le criticità incontrate.