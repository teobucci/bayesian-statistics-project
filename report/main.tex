\documentclass{article}

\newcommand{\coursename}{Bayesian Statistics}
\newcommand{\coursecode}{052499}
\newcommand{\coursesupervisor}{Alessandro Colombi}
\newcommand{\courseprof}{Prof. A. Guglielmi}
\newcommand{\papertitle}{Stochastic Block Model Prior with Ordering Constraints for Gaussian Graphical Models}

\usepackage[british]{babel} % british per avere l'access date nella bibliografia come i comuni mortali
\usepackage[utf8]{inputenc}
\usepackage[margin=1in]{geometry}
\usepackage{amsmath,amsthm,amsfonts,amssymb}
\usepackage{lipsum}
\usepackage{bm,bbm}
\usepackage{siunitx}
\setcounter{secnumdepth}{6} 
%\usepackage[toc]{appendix}
\usepackage{multirow}

\usepackage{booktabs}
\usepackage{multirow}
\usepackage{xcolor}
\usepackage{colortbl}

\usepackage{multicol}
\usepackage[none]{hyphenat}
\usepackage[small]{titlesec}
\usepackage{minted}
\usemintedstyle{default} % https://pygments.org/styles/

% Bibliography
\usepackage{csquotes}% Recommended
\usepackage[
    style=authoryear,
    url=false,
    firstinits=true,
    sorting=none,
    maxbibnames=99,
    natbib
    ]{biblatex}
\addbibresource{../bibliography.bib}% Syntax for version >= 1.2
% use \cite or \parencite

\usepackage{import}
\usepackage{pdfpages}
\usepackage{transparent}
\usepackage{xcolor}
\usepackage{graphicx}
\graphicspath{ {./images/} } % Path relative to the main .tex file
\usepackage{float}
\usepackage[font=footnotesize]{caption}
\usepackage{booktabs}

\usepackage{fontawesome}
\usepackage{authblk}

\usepackage{bookmark}% loads hyperref too
    \hypersetup{
        %pdftitle={Fundamentos de C\'alculo},
        %pdfsubject={C\'alculo diferencial},
        bookmarksnumbered=true,
        bookmarksopen=true,
        bookmarksopenlevel=1,
        hidelinks,% remove border and color
        pdfstartview=Fit, % Fits the page to the window.
        pdfpagemode=UseOutlines, %Determines how the file is opening in Acrobat; the possibilities are UseNone, UseThumbs (show thumbnails), UseOutlines (show bookmarks), FullScreen, UseOC (PDF 1.5), and UseAttachments (PDF 1.6). If no mode if explicitly chosen, but the bookmarks option is set, UseOutlines is used.
    }

\usepackage[acronym]{glossaries}
\setacronymstyle{long-short} 
\makenoidxglossaries
\newacronym{ggm}{GGM}{Gaussian Graphical Models}
\newacronym{sbm}{SBM}{Stochastic Block Model}
\newacronym{vi}{VI}{Variation of Information}

\setlength{\marginparwidth}{3.4cm}

%#########################################################

\title{
    \begin{figure}[htpb]
        \centering
        \includegraphics[scale=0.2]{images/logo-polimi}
    \end{figure}
    \normalfont \normalsize 
    \textsc{MSc. in Mathematical Engineering A.Y. 2022/2023\\ 
    Project Report of \coursename\ (\coursecode) -- \courseprof \\
    Supervisor: \coursesupervisor} \\
    [10pt] 
    \rule{\linewidth}{0.5pt} \\ [6pt] 
    \huge \papertitle \\
    \rule{\linewidth}{2pt}  \\ [10pt]
}

%\author{Teo Bucci, Filippo Cipriani, Filippo Pagella,\\ Flavia Petruso, Andrea Puricelli, Giulio Venturini}
%\author{T. Bucci, F. Cipriani, F. Pagella,\\ F. Petruso, A. Puricelli, G. Venturini}
%\author{T. Bucci\footnote{teo.bucci@mail.polimi.it, 10621873}, F. Cipriani\footnote{filippo.cipriani@mail.polimi.it, 10596877}, F. Pagella\footnote{filippo.pagella@mail.polimi.it, 10616351},\\ F. Petruso\footnote{flavia.petruso@mail.polimi.it, 10544566}, A. Puricelli\footnote{andrea3.puricelli@mail.polimi.it, 10632135}, G. Venturini\footnote{giulio.venturini@mail.polimi.it, 10624098}}

%\affil{\texttt{\{teo.bucci, filippo.cipriani, filippo.pagella, flavia.petruso, andrea3.puricelli, giulio.venturini\}@mail.polimi.it}}

\makeatletter
\renewcommand\AB@affilsepx{, \protect\Affilfont}
\makeatother
\author[1]{Teo Bucci}
\author[2]{Filippo Cipriani}
\author[3]{Filippo Pagella}
\author[4]{Flavia Petruso}
\author[5]{Andrea Puricelli}
\author[6]{Giulio Venturini}
\affil[1]{10621873}
\affil[2]{10596877}
\affil[3]{10616351}
\affil[4]{10544566}
\affil[5]{10632135}
\affil[6]{10624098}

\date{\normalsize \today}

\begin{document}

\maketitle

\begin{abstract} %in progress - per ora non lo manderei
Gaussian graphical models are used to study the conditional dependence structure among variables through the presence or absence of edges in the underlying undirected graph. In many applications, the variables can be grouped so that the graph to be learnt from the data has a block structure. Stochastic block models offer a powerful tool to detect such structure in a network. The goal of this project is to propose a new flexible prior that accounts for a random partition of the nodes, respects their ordering constraints and allows to learn a block-structured graph.

\vspace*{0.5cm}

\begin{center}
    The source code of the entire project,\\
    including this report and the presentations, is available at\\
    \faGithub\ \url{https://github.com/teobucci/bayesian-statistics-project}
\end{center}
\end{abstract}

\clearpage

\tableofcontents

%\setlength{\columnsep}{0.8cm}
%\begin{multicols}{2}
    %!TEX root = ./provabeamer.tex


\section{Background}

%Gaussian graphical models-------------------------------
\subsection{Gaussian Graphical Models}
\begin{frame}{Gaussian Graphical Models}

    Models for the \alert{conditional dependence structure} among variables, represented through an undirected graph $G=(V,E)$
    \begin{align*}
    \bm{x}_{1}, \ldots, \bm{x}_{n} \mid \bm{K} &\iid N_{p}(\bm{0},\bm{K}^{-1}), \bm{K}=\Sigma^{-1}, \bm{K} \in \mathbb{R}^{p\times p}  %%%%add bold
    \end{align*}
Conditional independence described through a \alert{map} between a \alert{graph} and a family of multivariate \alert{probability models}
\begin{align*}
Y_{i}\indep Y_{j} \mid Y_{-(ij)} \iff (i,j) \notin E \iff k_{ij}=0
\end{align*}
Usual prior for $\bm{K}$ conditionally to the graph is a G-Wishart
\begin{align*}
    \bm{K} \mid G &\iid \GWish(B,d)
\end{align*}
$G$ is a r.v. in the space of undirected graphs with $p$ nodes.\\
If we assume a possible \alert{grouping} of the variables, we need a prior $\pi(G)$ that induces a \alert{block structure on its adjacency matrix}.
\end{frame}


%%TO DO @Teo 
%Add bold sigma
%Add the image - STAVO PENSANDO, MA SE NE METTESSIMO DUE? UNA CON UN GRAFO CON ELEMENTI DA 1 A P STRUTTURA E UNA TABELLA - OK CHE E' SIMILE A QUELLO DI COLOMBI BUT STILL
%%

 

%Stochastic Block Models -----------------------
\subsection{Stochastic Block Models}
\begin{frame}{Stochastic Block Models for the prior on $G$}

Given a random network of data, SBM \alert{infere a node partition} based on similarity of connectivity patterns. Let
\begin{itemize}
    \item $H$ be the fixed number of clusters
    \item $\bm{z} \in \mathbb{R}^p$ be the vector of group memberships, $z_{i} \in \{1,\ldots,H\}$
\end{itemize} 

Then the model for the prior of $G$ conditionally on $\bm{z}$ is the following
\begin{align*}
    G_{ij} \mid \bm{z} &\ind\  f_{G}(G \mid \bm{z}) , \quad 1 \le i < j \le p \\ %alternativr G_{ij} \in {Edges}
    \bm{z} &\iid\ f_{z}(\bm{z})
\end{align*}
Where $f_{G}(G \mid \bm{z})$ and $f_{z}(\bm{z})$ can be chosen from different parametric families
\pause
\begin{center}
    Critical task: \alert{identification of the number of clusters $H$}
\end{center}

\end{frame}







\begin{frame}{Choosing the number of clusters $H$}

$H$ is usually estimated \alert{before} running the model. However
\begin{itemize}
    \item uncertainty at this stage is ignored (\alert{error quantification})
    \item the model does not account for new clusters (\alert{prediction})
\end{itemize}
\vspace*{0.5cm}
\pause
\alert{Extended stochastic block models (ESBM)}\\
Use priors for $\bm{z}$ that naturally allow to adaptively modify the number of groups $H$, \emph{e.g.},
\begin{itemize}
    \item Finite Dirichlet
    \item Infinite Dirichlet
    \item Mixture of Finite Mixtures
\end{itemize}

\pause
\begin{center}
    \alert{Most of the ESBM models do not pose constraints\\
    on the ordering of the nodes when generating the partition,\\
    that may be of relevance in real-life context (i.e., gene expression).}    
\end{center}


\end{frame}








%Changepoint models------------------
\subsection{Changepoint Models}
\begin{frame}{Changepoint Models}
    %\fg{0.4}{changepoint}
    Consider a process where at certain \alert{changepoints} the underlying generating mechanism changes. Let $\bm{x}=(x_{1},\ldots,x_{n})$ be \alert{ordered} observations each depending on a $\vartheta_{i}.$
    \[
        f(x_{1},\ldots,x_{n} \mid \vartheta_1, \vartheta_2, \ldots, \vartheta_{k+1}, \bm{z})=\prod_{j=1}^{k+1} \prod_{i=\tau_{j-1}+1}^{\tau_j} f(x_i \mid \vartheta_j)
    \]
    where $\tau_j$ are changepoint times and $z_{i} = 1$ if it's a changepoint, $0$ otherwise.
    We borrow the concept of \alert{generating a random partition with ordering constraints}:\footnote{$\vartheta$ can be marginalized out.}
    \begin{align*}
        \pi(\bm{z}, \bm{\vartheta} \mid \bm{y}) & \propto f(\bm{y} \mid \bm{\vartheta}, \bm{z}) \pi(\bm{\vartheta} \mid \bm{z}) \pi(\bm{z}) \\
        &=\left(\prod_{j=1}^{k+1} \prod_{i=\tau_{j-1}+1}^{\tau_j} f(y_i \mid \vartheta_j)\right)\left(\prod_{j=1}^{k+1} \pi(\vartheta_j)\right) \pi(\bm{z})
    \end{align*}
\end{frame}








%----------------------------------------
% Project goals
%-------------------------------------
\section{Project goals and next steps}


\begin{frame}{Goal of the project and next steps}

\alert{Goal}

Propose a \alert{new prior} that accounts for
%a random partition of the nodes, respects their ordering constraints and allows to learn a block structured graph
\begin{itemize}
    \item Ordering Constraint: taking advantage of the study of changepoint models
    %\item Random partition: using nonparametric prior on ordered partitions
    \item Block Structure: using a stochastic block model prior
\end{itemize}
\vspace*{0.5cm}
\pause
\alert{Next steps}
\begin{itemize}
    \item understanding and implementing the sampling strategy;
    \item using a nonparametric prior on ordered partitions.
\end{itemize}

\end{frame}

\begin{frame}{Main references}
    % GGM
    \nocite{colombiLearningBlockStructured2022a}
    \nocite{mohammadiBayesianStructureLearning2015a}
    % SBM
    \nocite{legramantiExtendedStochasticBlock2022}
    % Changepoint
    \nocite{bensonAdaptiveMCMCMultiple2018}
    \nocite{martinezNonparametricChangePoint2014}
    
    
    %biblatex
    %\printbibliograph
    %\renewcommand*{\bibfont}{\small}
    %bibtex
    \bibliographystyle{plain} % We choose the "plain" reference style
    \bibliography{bibliography} % Entries are in the refs.bib file
\end{frame}

\begin{frame}[plain]
    % Add background to content page
    \AddToShipoutPictureFG*{\includegraphics[width=\paperwidth]{Images/background.pdf}}
    \vspace*{1.2cm}
    \hspace*{1cm}{\Large Thank you!}\\
    \vspace*{0.6cm}
    \pause
    \hspace*{1cm}{\Huge \alert{Any questions?}}
\end{frame}


















%tenere?
\section*{Extra}
%Stochastic Block Models -----------------------
\begin{frame}{Stochastic Block Models for the prior on G}

\begin{align*}
    G_{ij} \mid \Theta,\bm{z} &\iid \Be(\theta_{ij}),  G \in \mathbb{R}^{p \times p}, \quad \theta_{ij} = \Theta_{z_{i}z_{j}}, \quad 1 \le i < j \le p \\ 
    \Theta_{z_{i}z_{j}} &\iid \Beta(\alpha, \beta)\\
    \bm{z} &\iid \pi_{z}(\bm{z}), \quad \bm{z} \in \mathbb{R}^p
\end{align*}
$\Theta$ can be marginalized out via beta-binomial conjugacy.
%Tattica: farei comparire la figura con un tap, e intanto si commenta che "per brevità non entriamo nel dettaglio ma se sono interessati si"
\fg{0.9}{sbm}
%Aggiungere da dove viene la caption
 %% Critical task: \alert{identification of the number of clusters H}
{\scriptsize From ``Colombi et al. Learning block structured graphs in Gaussian graphical models.''}
\end{frame}



%\end{multicols}

% USE NOCITE TO ADD SOURCES TO THE BIBLIOGRAPHY WITHOUT SPECIFICALLY CITING THEM IN THE DOCUMENT
%\nocite{zhixiong_modelling_2015}
%\nocite{*}

% \begin{minted}{R}
% fib <- function(n) {
%   if (n < 2)
%     n
%   else
%     fib(n - 1) + fib(n - 2)
% }
% fib(10) # => 55
% \end{minted}
% 
% \begin{minted}{R}
% # Creating a Graph
% attach(mtcars)
% plot(wt, mpg)
% abline(lm(mpg~wt))
% title("Regression of MPG on Weight")
% \end{minted}

% Optional, comment if not needed
\printnoidxglossary[type=\acronymtype, title=Glossary, toctitle=Glossary, numberedsection=autolabel]

\nocite{*}
\printbibliography[heading=bibintoc,title={References}]

\end{document}
