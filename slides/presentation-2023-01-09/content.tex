%!TEX root = ./main.tex

%-------------------------------------
%-------------------------------------
% The model
%-------------------------------------
%-------------------------------------
\section{The model}

%\subsection{The model}

\begin{frame}{Our model}

\alert{Goal}: given a set of n data with p variables, infer the block structure of their variables
without knowing the number of groups.

\pause

\alert{Additional constraints}: 
\begin{itemize}
    \item use a Stochastic Block Model    
    \item the partition must respect the original order of the variables
\end{itemize} 

\pause

\alert{The model}: 
\begin{align*}
    \bm{y}_1, \ldots, \bm{y}_n \mid \bm{K} & \iid \mathcal{N}_p(\mathbf{0}, \bm{K}^{-1} ) \\
    \bm{K} \mid G & \sim \GWish(b, D)\\
    P((i,j)&\in E\mid \bm{z},Q) = Q_{z_{i} z_{j}},\,\text{independent}\\
        Q_{rs} \mid \bm{z} &\ind \Beta(a,b), 1\leq r\leq s\leq H\\
    \rho_p & \sim \mathcal{L}\left(\rho_p\right)
\end{align*}  

\end{frame}


%-------------------------------------
%-------------------------------------
% The sampling strategy
%-------------------------------------
%-------------------------------------
\section{Sampling strategy}

\subsection{Overview of the sampling strategy}
\begin{frame}{Gibbs sampler}

formula del Gibbs sampler



\end{frame}




%-------------------------------------
% BDgraf
%-------------------------------------
\subsection{Updating the graph}
\begin{frame}{Birth and death algorithm for updating the graph}

   slide su bdgraph

\end{frame}


%-------------------------------------
% Updating the current partition
%-------------------------------------
\subsection{Updating the partition}


\begin{frame}{General steps for updating the partition}

   slide su two-step move to update the partition: 
   add/delete move (with general formula, no details) + shuffle

\end{frame}




\begin{frame}{Proposal ratio}

   slide su proposal ratio

\end{frame}




\begin{frame}{Adaptive step}

   non ho ben capito , vuole solo una slide sulla parte adattiva?
   forse è un po' troppo breve per una slide sola

\end{frame}



%-------------------------------------
%-------------------------------------
% Next steps
%-------------------------------------
%-------------------------------------
\section{Next steps}
\begin{frame}{Next steps}

    \begin{itemize}
        \item Praying that the code works
    \end{itemize}



\end{frame}



%----------------------------------------
% References
%-------------------------------------

\begin{frame}{Main references}
    % GGM
    \nocite{colombiLearningBlockStructured2022a}
    \nocite{mohammadiBayesianStructureLearning2015a}
    % SBM
    \nocite{legramantiExtendedStochasticBlock2022}
    % Changepoint
    \nocite{bensonAdaptiveMCMCMultiple2018}
    \nocite{martinezNonparametricChangePoint2014}
    
    
    %biblatex
    \printbibliography
    \renewcommand*{\bibfont}{\small}
    %bibtex
    %\bibliographystyle{plain} % We choose the "plain" reference style
    %\bibliography{bibliography} % Entries are in the refs.bib file
\end{frame}



\begin{frame}[plain]
    % Add background to content page
    \AddToShipoutPictureFG*{\includegraphics[width=\paperwidth]{Images/background.pdf}}
    \vspace*{1.2cm}
    \hspace*{1cm}{\Large Thank you!}\\
    \vspace*{0.6cm}
    \hspace*{1cm}{\Huge \alert{Any questions?}}
\end{frame}





%tenere?
\section*{Extra}

\begin{frame}{Extra content}


\end{frame}


