%!TEX root = ./main.tex

%-------------------------------------
%-------------------------------------
% The model
%-------------------------------------
%-------------------------------------
\section{The model}

%\subsection{The model}

\begin{frame}{Our model}

\alert{Goal}: given a set of $n$ data with $p$ variables, infer the block structure of their variables
without knowing the number of groups.

\pause

\alert{Additional constraints}: 
\begin{itemize}
    \item use a Stochastic Block Model    
    \item the partition must respect the original order of the variables
\end{itemize} 

\pause

\alert{The model}: 
\begin{align*}
    \bm{y}_1, \ldots, \bm{y}_n \mid \bm{K} & \iid \mathcal{N}_p(\mathbf{0}, \bm{K}^{-1} ) \\
    \bm{K} \mid G & \sim \GWish(b, D)\\
    P((i,j)&\in E\mid \bm{z},Q) = Q_{z_{i} z_{j}},\,\text{independent}\\
        Q_{rs} \mid \bm{z} &\ind \Beta(a,b), 1\leq r\leq s\leq H\\
    \rho_p & \sim \mathcal{L}\left(\rho_p\right)
\end{align*}

% TODO inserire la prior?

\end{frame}


%-------------------------------------
%-------------------------------------
% The sampling strategy
%-------------------------------------
%-------------------------------------
\section{The sampling strategy}

\subsection{Overview of the sampling strategy}
\begin{frame}{Block Gibbs Sampler}
    Full conditional distributions for our model:
    \begin{itemize}
        \item Precision Matrix: $P(\bm{K} \mid \bm{Y},G,\bm{z}) \propto P(\bm{Y} \mid \bm{K})P(\bm{K} \mid {G})$ 

        \item Graph: $P(G \mid \bm{Y},\bm{K},\bm{z}) \propto P(\bm{Y} \mid \bm{K})P(\bm{K} \mid {G})P(G \mid \bm{z})$ 

        \item Random Partition: $P(\bm{z} \mid \bm{Y},\bm{K},G) \propto P(\bm{Y} \mid \bm{K})P(\bm{K} \mid {G})P(G \mid \bm{z})P(\bm{z}) \propto P(G \mid \bm{z})P(\bm{z}) $
    \end{itemize}

    \pause 

    We implement a Block Gibbs-Sampler strategy:
    \begin{enumerate}
        \item \alert{Sampling Graph and Precision Matrix}\\
        $G$ and $\bm{K}$ - given $\bm{z}$ - are sampled using a modified version of a Birth-and-Death chain, changing one link at a time
        \item \alert{Sampling the Random Partition}\\
        We can sample $\bm{z}$ through an adaptive MCMC\vphantom{changepoint} sampler, working conditionally on $G$ after the graph update
    \end{enumerate}
\end{frame}




%-------------------------------------
% BDgraph
%-------------------------------------
\subsection{Updating the graph}
\begin{frame}{Birth and death algorithm for updating the graph}

    \texttt{BDGraph} is an algorithm that follows a Birth-and-Death approach to decide whether to \alert{add} a new edge to the graph or \alert{delete} an already existing one.

    How did we change it?

    \pause
    
    % https://people.inf.ethz.ch/markusp/teaching/guides/guide-tables.pdf
    \begin{table}[tb]
        \centering
        \begin{tabular}{lcc}
        \toprule
        & Conditional & B/D rates \\ % TODO Full conditional non è il termine giusto, join conditional?
        \hline
        \textbf{Before} & $P(G,\bm{K} \mid \bm{Y}) \propto \mathcal{L}(\bm{Y} \mid \bm{K})\mathcal{L} (\bm{K} \mid {G})P(G)$ & $\frac{P(G')}{P(G)}$ \\
        \textbf{After}  & $P(G,\bm{K} \mid \bm{Y}, \textcolor{sleekRed}{\bm{z}}) \propto \mathcal{L}(\bm{Y} \mid \bm{K})\mathcal{L}(\bm{K} \mid {G})\textcolor{sleekRed}{P(G \mid \bm{z})}$ & $\frac{P(G' \mid \textcolor{sleekRed}{\bm{z}})}{P(G \mid \textcolor{sleekRed}{\bm{z}})}$ \\
        \bottomrule
        \end{tabular}
    \end{table}
    where $G' = G^{\pm e}$ and $e$ is an edge.

    \pause

    We computed the birth and death rates as:
    \[
        \text{Birth rate} = \frac{P(G^{+ e})}{P(G)} = \frac{S_{uv} + a}{T_{uv} - S_{uv} + b} \qquad \text{Death rate} =\frac{P(G^{- e})}{P(G)} = \frac{T_{uv} - S_{uv} + b}{S_{uv} + a}
    \]
    % TODO l'utente non sa cosa siano T,S,a,b forse si potrebbe balzare
    
\end{frame}


%-------------------------------------
% Updating the current partition
%-------------------------------------
\subsection{Updating the partition}


\begin{frame}{General steps for updating the partition}
    
    We perform an (adaptive) \alert{split and merge} as in \cite{bensonAdaptiveMCMCMultiple2018}.
    \begin{enumerate}
        \item With probability $\alpha_{\text{split}}$, usually $0.5$, choose an \alert{split move (add)}, otherwise a \alert{merge move (delete)}.
        \begin{enumerate}
            \item Propose a new partition by splitting one group into two or merging two adjacent ones.
            \item Accept or reject using Metropolis Hastings. The target is:
            $f(\rho \mid G) \approx f_G(G \mid \rho) f_{\rho}(\rho)$
            \[
                \alpha_{\text{accept,merge}} = \min % TODO non sono sicuro che qua ci vada del/merge, nelle note non aggiornate c'era add
                \bigg\{1,
                \underbrace{\frac{f_G\left(G \mid \rho'\right)}{f_G(G \mid \rho)}}_{\substack{\text{likelihood}\\\text{ratio}}}
                \underbrace{\frac{f_\rho\left(\rho'\right)}{f_\rho(\rho)}}_{\substack{\text{prior}\\\text{ratio}}}
                \underbrace{\frac{P(\text{choose merge})}{P(\text{choose split})} \frac{P(\text{merge})}{P(\text{split})}}_{\text{proposal ratio}}
                \bigg\}
            \]
        \end{enumerate}
        \item To improve the mixing of the chain we also perform a \alert{shuffle move}.
        \begin{enumerate}
            \item Propose a new partition by moving some nodes from a group to an adjacent one.
            \item Accept it using Metropolis Hastings
        \end{enumerate}

    \end{enumerate}

\end{frame}

\begin{frame}{Visual explanation of the possible partition updates}
    \fg{1}{update_partition.pdf}
\end{frame}








\begin{frame}{Proposal ratio (case merge)}

\begin{enumerate}
\item We introduce $a^{(t)}=(a_{1}, ... , a_{n_{nodes}})$ and $d^{(t)}=(d_{1}, ... , d_{n_{nodes})}$ the vectors of weights to choose the node where to perform or the split or the merge 

\item Compute $a^{*} = \sum_{n_{j}\neq 0}{a_{j}^{(t)}}$  and $d^{*} = \sum_{n_{j}\neq 0}{d_{j}^{(t)}}$

\item Extract the element $i$ from the admissible nodes with P(choosing i) $\varpropto \frac{d_{i}^{(t)}}{d^{*}}$

\item 
Proposal ratio = $\frac{P(\text{choose merge})}{P(\text{choose split})} \frac{P(\text{merge at node $i$})}{P(\text{split at node $i$})}$

  %TODO MAKE THEM BIGGER! 


\end{enumerate}

\pause 

\[ 
\text{Proposal ratio} =
    \begin{cases} 
         0 & \text{there is only one group (no merge possible)}  \\
         \frac{1}{2}\frac{d_{i}^{(t)}}{d^{*}} & \text{there are as many groups as nodes (only merge possible)} \\
         \frac{d_{i}^{(t)}}{d^{*}} \frac{a_{i}^{(t)}}{a^{*}+a_{i}} & \text{general case} \\
    \end{cases}
\]

\end{frame}




\begin{frame}{Adaptive step}
    To select the position where to perform the chosen move (add or delete), we use two vectors of weights, respectively $\bm{a}^{(t)}$
    and $\bm{d}^{(t)}$. \\
    These weights are not constant over time iterations, but follow, as in \cite{bensonAdaptiveMCMCMultiple2018}, the following 
    \textbf{adaptation scheme}.\\

    At iteration $t$ :
    \begin{enumerate}
        \item If an $add$ move at point $i$ has been accepted then update only the $a_i^{(t)}$ parameter as follows
		\[
		\log (a_i^{(t+1)})=\log (a_i^{(t)})+\frac{h}{t / n}(\alpha_{\text{add}}-\alpha_{\text{target}}) .
		\]
        \item If a $delete$ move at point $i$ has been accepted then update only the $d_i^{(t)}$ parameter as follows
		\[
		\log (d_i^{(t+1)})=\log (d_i^{(t)})+\frac{h}{t / n}(\alpha_{\text{del}}-\alpha_{\text{target}}) .
		\]
        \end{enumerate}
$h$ - Initial Adaptation $(h>0)$,
$t / n$ - iterations $(t)$ per number of datapoints $(n)$

\end{frame}



%-------------------------------------
%-------------------------------------
% Next steps
%-------------------------------------
%-------------------------------------
\section{Next steps}
\begin{frame}{Next steps}

    \begin{itemize}
        \item Praying that the code works
    \end{itemize}



\end{frame}



%----------------------------------------
% References
%-------------------------------------

\begin{frame}{Main references}
    % GGM
    \nocite{colombiLearningBlockStructured2022a}
    \nocite{mohammadiBayesianStructureLearning2015a}
    % SBM
    \nocite{legramantiExtendedStochasticBlock2022}
    % Changepoint
    \nocite{bensonAdaptiveMCMCMultiple2018}
    \nocite{martinezNonparametricChangePoint2014}
    
    
    %biblatex
    \printbibliography
    \renewcommand*{\bibfont}{\small}
    %bibtex
    %\bibliographystyle{plain} % We choose the "plain" reference style
    %\bibliography{bibliography} % Entries are in the refs.bib file
\end{frame}



\begin{frame}[plain]
    % Add background to content page
    \AddToShipoutPictureFG*{\includegraphics[width=\paperwidth]{Images/background.pdf}}
    \vspace*{1.2cm}
    \hspace*{1cm}{\Large Thank you!}\\
    \vspace*{0.6cm}
    \hspace*{1cm}{\Huge \alert{Any questions?}}
\end{frame}





%tenere?
\section*{Extra}

\begin{frame}{Extra content}


\end{frame}


