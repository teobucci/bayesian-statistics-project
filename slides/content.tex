%!TEX root = ./main.tex


\section{Background}

%Gaussian graphical models-------------------------------
\subsection{Gaussian Graphical Models}
\begin{frame}{Gaussian Graphical Models}
    Models used to learn the \alert{conditional dependence structure} among variables, represented through an undirected graph G = (V,E)
\begin{columns}
\begin{column}{0.5\textwidth}

\newcommand{\iid}{\overset{\mathrm{iid}}{\sim}}

\centering
\begin{align*}
    y_{1}, \ldots, y_{n} \mid \mathbf{K} \iid N_\mathbf{{p}}(0,\mathbf{K}), \mathbf{K}=\Sigma^{-1}\\ 
    \mathbf{K} \mid G \iid G-Wishart(B,d) ,  K \in \mathbb{R}^{p\times p}\\
    G_{ij} \mid z \iid ?\\ 
    z \iid ?
\end{align*}
\end{column}

\begin{column}{0.5\textwidth}
Qui metterei una immagine acattivante (tabella stile slides Colombi?) dei gaussian graphical models dove si capisce che ogni singola entry della adjacency matrix modella la dipendenza tra le componenti
\end{column}

\end{columns}
\centering
How to introduce an appropriate prior $\pi_{z}$ for \alert{partition generation} and a model for the \alert{edge generation} conditionally on z? 
\end{frame}











%Stochastic Block Models -----------------------
\subsection{Stochastic Block Models}

\begin{frame}{Stochastic Block Models}
    Statistical models for \alert{community detection}\\     
    Goal: given a random network of data, infere a node partition by clustering nodes within groups sharing similar connectivity patterns


    \begin{columns}
        \begin{column}{0.30\textwidth}
            \begin{align*}
                G_{ij} \mid \Theta,z \overset{\mathrm{iid}}{\sim} Be(\theta_{ij}),  G \in R^{p×p}, \\
                \theta_{ij} = \Theta_{z_{i}z_{j}}, 1 \le i < j \le p \\ 
                \Theta_{rs} \overset{\mathrm{iid}}{\sim} Beta(\alpha, \beta)\\
                z \overset{\mathrm{iid}}{\sim} \pi_{z}(z), z \in R^p
            \end{align*}
            \end{column}
        \begin{column}{0.30\textwidth}
            \alert{Adjacency matrix G}
            \begin{table}[htpb]
            %\setlength{\arrayrulewidth}{1pt}
            \centering
            \begin{tabular}{c|c|c|c|c|c|} 
            \hline
            1   & &    &    &   &     \\ 
            \hline
            2  &  $y_{21}$ &    &    &   &     \\ 
            \hline
            ...  &   &    &    &   &     \\ 
            \hline
            ...  &   &    &    &   &     \\ 
            \hline
            N  & $y_{n1}$   &    &    &   &  
            \end{tabular}
            \end{table}
        \end{column}


        \begin{column}{0.25\textwidth}
        %\centering
        \alert{Cluster matrix $\Theta$}

        Inserire come nelle note mie e di Andre, sistemare anche la adjacency matrix
        \end{column}

    \end{columns}



    \begin{center}
    Critical task: \alert{identification of the number of clusters H}
    \end{center}


\end{frame}















\begin{frame}{Choosing the prior for the number of clusters}

\begin{columns}

\begin{column}{0.5\textwidth} 

    Number of clusters H usually estimated \textit{before} running the model
    \begin{itemize}
        \item prior knowledge
        \item BIC
        \item Cross-validation methods
    \end{itemize}
\end{column}

\begin{column}{0.5\textwidth}
    Problems: underestimation of model uncertainty (?). prediction: impossible to classify elements in a new cluster
\end{column}
\end{columns}

\alert{Extended stochastic block models (ESBM)}\\  
Use priors for z that naturally allow to increase the number of groups, e.g., 
    \begin{itemize}
        \item Chinese restaurant process (CRP)
        \item Mixture of finite mixtures (MFM)
    \end{itemize}

Most of the ESBM models do not pose constraints\\ on the \alert{ordering of the elements} in the partition
\end{frame}



%Changepoint models------------------
\subsection{Changepoint Models}
\begin{frame}{Changepoint Models}

Changepoint models allow generations of random partitions that with ordering constraints (or sth like this)

qui nei changepoints inserirei la prima formula sui changepoints 
per fare vedere cosa sono, e poi forse l'algoritmo per anew e dnew stile Metropolis Hastings, insieme a una frase sulla loro utilità contestualizzata alla generazione di partizioni
forse ci spammerei anche la bellissima prova di Martinez che avete dimostrato , SO CHE ALLA GUGLIELMI PIACE


\end{frame}


%----------------------------------------
% Project goals
%-------------------------------------
\section{Project goals}


\begin{frame}[containsverbatim]{Goal of the project}
Propose a \alert {new flexible prior that accounts for a random partition of the nodes}, respects their ordering constraints and allows to learn a block structured graph

 \begin{itemize}
     \item Ordering Constraint: borrow ideas from changepoint models
     \item Random partition: nonparametric prior on ordered partitions
     \item Block Structure: use a stochastic block model prior
 \end{itemize}


Le frasi sopra sono copiate dalla presentazione di colombi.

Secondo me se riuscissimo sarebbe figo mettere un qualche schemino  Dove specifichiamo concettualmente come i tre aspetti del background si mettono insieme, 
e anche (ma forse è troppo) la formula completa di Colombi dove il Gaussian Graphical model ha come probabilità condizionata sul grafo e come prior di z lo stochastic block model

\end{frame}

\section{Next steps}
\begin{frame}

forse dovremmo capire meglio quali sono i next steps

\begin{itemize}
    \item sampling strategy
    \item simulation strategy
\end{itemize}


Andrà descritto il Gibbs sampler, andrà detto il termine "Monte Carlo", un pizzico di "full conitional" e sembreremo Bayesiani DOC

\centering \Huge FAKE IT TILL YOU MAKE IT
\end{frame}





\begin{frame}{Main References}
       \nocite{*}
        \bibliography{bibliography}
        \bibliographystyle{plain}
\end{frame}

\section*{Questions}


\begin{frame}{Extra slide 1}
metterei le formule di 
\begin{itemize}
    \item Dirichlet multinomial
    \item CRP
    \item Mixture of finite mixtures (not strictly necessary)

    

\end{itemize} 
\end{frame}



%\begin{frame}
%    \tableofcontents
%\end{frame}

\begin{frame}
    {Goal of the project}
    Consider a set $p$ of variables.\\
    Here's the turning point: we suppose that there is an underlying order between these variables.\\
    How do we learn the block structure?
\end{frame}

\begin{frame}{First step: Gaussian Graphical Models}
    asd
\end{frame}

\begin{frame}{Second step: Stochastic Block Models}
    asd
\end{frame}

\begin{frame}{Third step: Changepoint Models}
    as
\end{frame}

% \begin{frame}{All together}
%     \emph{``Are you getting it? These are not three separate devices: This is one device''}
%     \fg{1}{steve-mod}
% \end{frame}

\begin{frame}[fragile]
\frametitle{Python Code listing in Beamer}
The following Python code adds two numbers and display the result using \verb|print()| function:
\rule{\textwidth}{1pt}
\scriptsize
\begin{minted}{python}
    # This program adds two numbers
    num1 = 1.5
    num2 = 6.3
    # Add two numbers
    sum = num1 + num2
    # Display the sum
    print('The sum of {0} and {1} is {2}'.format(num1, num2, sum))
\end{minted}
\rule{\textwidth}{1pt}
\end{frame}


% Use starred version (e.g. \section*{Section name})
    % to disable (sub)section page.
    \section{Section 1}
    \subsection{Subsection 1.1}
    \begin{frame}{Simple frame}
        This is a simple frame.
    \end{frame}

    \begin{frame}[plain]{Plain frame}
        This is a frame with plain style and it is numbered.
    \end{frame}
    
    \subsection{Subsection 1.2}
    \begin{frame}[t]
        This frame has an empty title and is aligned to top.
    \end{frame}
    
    \begin{frame}[noframenumbering]{No frame numbering}
        This frame is not numbered and is citing reference \cite{knuth74}.
    \end{frame}
    
    \begin{frame}{Typesetting and Math}
        The packages \texttt{fontenc} and \texttt{FiraSans}\footnote{\url{https://fonts.google.com/specimen/Fira+Sans}}\textsuperscript{,}\footnote{\url{http://mozilla.github.io/Fira/}} are used to properly set the main fonts.
        \vfill
        This theme provides styling commands to typeset \emph{emphasized}, \alert{alerted}, \textbf{bold}, \dots
        \vfill
        \texttt{FiraSans} also provides support for mathematical symbols:
        \begin{align*}
            e^{i\pi} + 1 & = 0, \\
            \int_{-\infty}^\infty e^{-x^2}\,\mathrm{d}x & = \sqrt{\pi}.
        \end{align*}
    \end{frame}

    \section{Section 2}
    \begin{frame}{Blocks}
        \begin{block}{Block}
            Text.
        \end{block}
        \pause
        \begin{alertblock}{Alert block}
            Alert \alert{text}.
        \end{alertblock}
        \pause
        \begin{exampleblock}{Example block}
            Example
        \end{exampleblock}
    \end{frame}
    
    %\begin{frame}{Lists}
    %    \begin{columns}[t, onlytextwidth]
    %        \column{0.33\textwidth}
    %            Items:
    %            \begin{itemize}
    %                \item Item 1
    %                \begin{itemize}
    %                    \item Subitem 1.1
    %                    \item Subitem 1.2
    %                \end{itemize}
    %                \item Item 2
    %                \item Item 3
    %            \end{itemize}
    %        
    %        \column{0.33\textwidth}
    %            Enumerations:
    %            \begin{enumerate}
    %                \item First
    %                \item Second
    %                \begin{enumerate}
    %                    \item Sub-first
    %                    \item Sub-second
    %                \end{enumerate}
    %                \item Third
    %            \end{enumerate}
    %        
    %        \column{0.33\textwidth}
    %            Descriptions:
    %            \begin{description}
    %                \item[First] Yes.
    %                \item[Second] No.
    %            \end{description}
    %    \end{columns}
    %\end{frame}
\setbeamertemplate{caption}[numbered]
    \begin{frame}{Figures and Tables}
        \begin{columns}
            \column{0.4\textwidth}
                \begin{figure}
                    \centering
                    \includegraphics[scale=0.15]{../report/images/logo-polimi-transparent}
                    \caption{Figure caption.}
                    \label{fig:focuslogo}
                \end{figure}
                
            \column{0.6\textwidth}
                \begin{table}
                    \centering
                    \begin{tabular}{rcc}
                         & Heading 1 & Heading 2 \\\hline
                        Row 1 & \(v_{11}\) & \(v_{12}\) \\
                        Row 2 & \(v_{21}\) & \(v_{22}\) \\
                        Row 3 & \(v_{31}\) & \(v_{32}\) \\
                    \end{tabular}
                    \caption{Table caption.}
                    \label{tab:demo}
                \end{table}
        \end{columns}
    \end{frame}

    \appendix
    \begin{frame}{References}
        \nocite{*}
        \bibliography{bibliography}
        \bibliographystyle{plain}
    \end{frame}
    
    \begin{frame}{Backup frame}
        This is a backup frame, useful to include additional material for questions from the audience.
        \vfill
        The package \texttt{appendixnumberbeamer} is used not to number appendix frames.
    \end{frame}

\begin{frame}
    Thank you!
\end{frame}