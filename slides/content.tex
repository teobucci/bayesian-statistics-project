%!TEX root = ./main.tex


\section{Background}

%Gaussian graphical models-------------------------------
\subsection{Gaussian Graphical Models}
\begin{frame}{Gaussian Graphical Models}

    Models for the \textbf{conditional dependence structure} among variables, represented through an undirected graph $G=(V,E)$
    \begin{align*}
    \bm{x}_{1}, \ldots, \bm{x}_{n} \mid \bm{K} &\iid N_{p}(\bm{0},\bm{K}^{-1}), \bm{K}=\Sigma^{-1}, \bm{K} \in \mathbb{R}^{p\times p}  %%%%add bold
    \end{align*}
Conditional independence described through a \textbf{map} between a \textbf{graph} and a family of multivariate \textbf{probability models}
\begin{align*}
Y_{i}\perp Y_{j} \mid Y_{-(ij)} \iff (i,j) \notin E \iff k_{ij}=0
\end{align*}
Usual prior for $\bm{K}$ conditionally to the graph is a G-Wishart
\begin{align*}
    \bm{K} \mid G &\iid \text{G-Wishart}(B,d)
\end{align*}
$G$ is a r.v. in the space of undirected graphs with p nodes.\\
If we assume a possible \textbf{grouping} of the variables, we need a prior $\pi(G)$ that induces a \textbf{block structure on its adjacency matrix}.
\end{frame}


%%TO DO @Teo 
%Add bold sigma
%Add the image - STAVO PENSANDO, MA SE NE METTESSIMO DUE? UNA CON UN GRAFO CON ELEMENTI DA 1 A P STRUTTURA E UNA TABELLA - OK CHE E' SIMILE A QUELLO DI COLOMBI BUT STILL
%%

 

%Stochastic Block Models -----------------------
\subsection{Stochastic Block Models}
\begin{frame}{Stochastic Block Models for the prior on $G$}

Given a random network of data, SBM \textbf{infere a node partition} based on similarity of connectivity patterns. Let
\begin{itemize}
    \item $H$ be the fixed number of clusters
    \item $\bm{z} \in \mathbb{R}^p$ be the vector of group memberships, $z_{i} \in \{1,\ldots,H\}$
\end{itemize} 

Then the model for the prior of $G$ conditionally on $\bm{z}$ is the following
\begin{align*}
    G_{ij} \mid \bm{z} &\ind\  f_{G}(G \mid \bm{z}) , \quad 1 \le i < j \le p \\ %alternativr G_{ij} \in {Edges}
    \bm{z} &\iid\ f_{z}(\bm{z})
\end{align*}
Where $f_{G}(G \mid \bm{z})$ and $f_{z}(\bm{z})$ can be chosen from different parametric families
\pause
\begin{center}
    Critical task: \alert{identification of the number of clusters H}
\end{center}

\end{frame}







\begin{frame}{Choosing the number of clusters $H$}

$H$ is usually estimated before using the model to update the prior knowledge. However
\begin{itemize}
    \item uncertainty at this stage is ignored (\textbf{error quantification})
    \item the model does not account for new clusters (\textbf{prediction})
\end{itemize}
\horiz
%%%%%%%%qui spunta un tap
\textbf{\alert{Extended stochastic block models (ESBM)}}\\  
Use priors for $\bm{z}$ that naturally allow to adaptively modify the number of groups $H$, \emph{e.g.},
\begin{itemize}
    \item Finite Dirichlet
    \item Infinite Dirichlet
    \item Mixture of Finite Mixtures (MFM)
\end{itemize}
\centering
Most of the ESBM models do not pose constraints\\ on the \textbf{ordering of the nodes} when generating the partition, that may be of relevance in real-life context (i.e., gene expression)
\end{frame}








%Changepoint models------------------
\subsection{Changepoint Models}
\begin{frame}{Changepoint Models}
    %\fg{0.4}{changepoint}
    Consider a process where at certain \textbf{changepoints} the underlying generating mechanism changes. Let $\bm{x}=(x_{1},\ldots,x_{n})$ be \textbf{ordered} observations each depending on a $\vartheta_{i}.$\\
    \[
        f(x_{1},\ldots,x_{n} \mid \vartheta_1, \vartheta_2, \ldots, \vartheta_{k+1}, \bm{z})=\prod_{j=1}^{k+1} \prod_{i=\tau_{j-1}+1}^{\tau_j} f(x_i \mid \vartheta_j)
    \]
    where $\tau_j$ are changepoint times and $z_{i} = 1$ if it's a changepoint, $0$ otherwise.
    We borrow the concept of \textbf{generating a random partition with ordering constraints}:\footnote{$\vartheta$ can be marginalized out.}
    \begin{align*}
        \pi(\bm{z}, \bm{\vartheta} \mid \bm{y}) & \propto f(\bm{y} \mid \bm{\vartheta}, \bm{z}) \pi(\bm{\vartheta} \mid \bm{z}) \pi(\bm{z}) \\
        &=\left(\prod_{j=1}^{k+1} \prod_{i=\tau_{j-1}+1}^{\tau_j} f(y_i \mid \vartheta_j)\right)\left(\prod_{j=1}^{k+1} \pi(\vartheta_j)\right) \pi(\bm{z})
    \end{align*}
\end{frame}








%----------------------------------------
% Project goals
%-------------------------------------
\section{Project goals and next steps}


\begin{frame}{Goal of the project}

Propose a \alert{new flexible prior that accounts for a random partition of the nodes}, respects their ordering constraints and allows to learn a block structured graph

 \begin{itemize}
     \item Ordering Constraint: borrow ideas from changepoint models
     \item Random partition: nonparametric prior on ordered partitions
     \item Block Structure: use a stochastic block model prior
 \end{itemize}

forse dovremmo capire meglio quali sono i next steps

\begin{itemize}
    \item sampling strategy
    \item simulation strategy
\end{itemize}

Backup slides: metterei le formule di 
\begin{itemize}
    \item Dirichlet multinomial
    \item CRP
    \item Mixture of finite mixtures (not strictly necessary)

\end{itemize} 
\end{frame}



\begin{frame}{Main references}
    % GGM
    \nocite{colombiLearningBlockStructured2022a}
    \nocite{mohammadiBayesianStructureLearning2015a}
    % SBM
    \nocite{legramantiExtendedStochasticBlock2022}
    % Changepoint
    \nocite{bensonAdaptiveMCMCMultiple2018}
    \nocite{martinezNonparametricChangePoint2014}
    
    \renewcommand*{\bibfont}{\small}
    \printbibliography
\end{frame}

\begin{frame}
    Thank you!
    \pause
    Questions?
\end{frame}
